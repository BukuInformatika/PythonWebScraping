\section{Membangun Web Scraper}
\subsection{BeautifulSoup}
\par Beautiful Soup adalah \textit{library} Python untuk menarik atau mengambil data dari file HTML dan XML. Hal ini bisa bekerja dengan parser apapun untuk menavigasi, mencari, dan memodifikasi pohon parse. Ini biasanya menghemat waktu menulis kode dan hari kerja. BeautifulSoup juga mencoba memahami yang tidak masuk akal; sedikit membantu memformat ulang dan mengatur web yang berantakan dengan memperbaiki HTML yang kacau dan menyajikan kepada kita struktur dalam Python yang mudah dibaca dan mewakili struktur XML.
\textbf{Menginstal BeautifulSoup}
Karena \textit{library} BeautifulSoup bukan Python \textit{library} default, maka harus terlebih dahulu diinstal. Kita akan menggunakan \textit{library} BeautifulSoup versi 4 (BS4).

untuk Linux:

\$ sudo apt-get install python-bs4

dan untuk Mac:

\$ sudo easy\_install pip

Menginstal \textit{package} di Windows hampir sama dengan proses untuk Mac dan Linux. Unduh rilisan terakhir BeautifulSoup 4, navigasikan ke direktori tempat megekstrak lalu jalankan:

instal setup.py python

BeautifulSoup sekarang akan dikenali sebagai \textit{library} Python di sistem. Kita masih bisa menguji ini dengan membuka terminal Python dan mengimportnya:

\begin{algorithm}

\$ python
bs4 import from BeautifulSoup
\end{algorithm}


\textbf{Menjalankan BeautifulSoup}

sebelum kita melompat pada kode, mari kita bahas terlebih dahulu hal-hal mendasar dari HTML, dan beberapa aturan dalam \textit{web scraping}
\begin{algorithm}
<!DOCTYPE html>  
<html>  
    <head>
    </head>
    <body>
        <h1> First Scraping </h1>
        <p> Hello World </p>
    <body>
</html>
\end{algorithm}
\begin{enumerate}

\item \verb|<! DOCTYPE html>:| Dokumen HTML harus dimulai dengan deklarasi tipe.
\item Dokumen HTML terdapat di antara \verb|<html>| dan \verb|</html>.|
\item Deklarasi meta dan skrip dari dokumen HTML berada di antara \verb|<head>| dan \verb|</head>.|
\item Bagian yang terlihat dari dokumen HTML adalah antara tag \verb|<body>| dan \verb|</body>.|
\item Judul judul didefinisikan dengan tag \verb|<h1>| hingga \verb|<h6>.|
\item Paragraf didefinisikan dengan tag \verb|<p>.|
\end{enumerate}

Tag berguna lainnya termasuk \verb|<a>| untuk hyperlink, \verb|<table>| untuk tabel, \verb|<tr>| untuk baris tabel, dan \verb|<td>| untuk kolom tabel. Juga, tag HTML kadang-kadang datang dengan atribut id atau kelas. Atribut id menentukan id unik untuk tag HTML dan nilainya harus unik dalam dokumen HTML. Atribut kelas digunakan untuk menentukan gaya yang sama untuk tag HTML dengan kelas yang sama. Kita dapat menggunakan id dan kelas ini untuk membantu kita menemukan data yang kita inginkan.