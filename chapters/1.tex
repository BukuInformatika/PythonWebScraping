Pada bab ini, kita akan menjalankan program Python yang pertama yaitu, hello\_world.py. Tahap pertama, kita akan memastikan apakah Python sudah terinstall dengan benar. Kita juga akan menginstall teks editor untuk menulis program Python.
\section{Menyiapkan \textit{Programming Environment}}
Di sini, kita akan melihat dua versi utama Python yang saat ini digunakan dan menguraikan langkah-langkah untuk menyiapkan Python pada sistem.
\subsection{Python 2 dan Python 3}
Saat ini, telah tersedia dua versi Python: Python 2 dan Python yang lebih baru, yaitu Python 3. Mengapa? Karena setiap bahasa pemrograman berevolusi ketika ide dan teknologi baru muncul atau berkembang, dan tentu saja para pengembang bahasa Python terus membuat Python agar lebih fleksibel dan kuat. Sebagian besar perubahan yang dilakukan, berkembang sedikit demi sedikit secara teratur dan hampir tidak terlihat, tetapi dalam beberapa kasus kode yang ditulis untuk Python 2 mungkin tidak berjalan dengan baik pada sistem yang menggunakan Python 3.
\subsection{Menjalankan Kode}
Python dilengkapi dengan \textit{interpreter} yang berjalan di jendela terminal, yang memungkinkan kita untuk mencoba kode Python tanpa harus menyimpan dan menjalankan seluruh program. Contohnya adalah seperti pada
\begin{lstlisting}[language=Python, label={lst:hello}, caption=Kode Pada Jendela Terminal]
>>> print("Hello World!")
Hello World!
\end{lstlisting}
Pada baris pertama, adalah baris yang kita tuliskan sendiri perintahnya, lalu bisa dieksekusi dengan cara menekan tombol enter. Sebagian besar contoh yang akan ditampilkan pada buku ini akan dijalankan melalui teks editor, karena kita akan menulis kode pada teks editor tersebut. Setiap kali Anda melihat tiga buah tanda kurung siku seperti pada \ref{lst:hello} itu artinya kita sedang mengunakan jendela terminal.