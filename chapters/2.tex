
\par \textbf{Apa Itu Web Scraping?}
\par Secara teori, pengikisan web adalah praktik mengumpulkan data melalui cara apa pun selain program yang berinteraksi dengan API (atau melalui manusia menggunakan browser). Cara ini paling umum dilakukan dengan menulis program otomatis yang membutuhkan server web, meminta data (biasanya dalam bentuk HTML, Json, dan file lain yang terdiri dari halaman web), dan kemudian mem-\textit{parsing} data tersebut untuk mengekstrak informasi yang diperlukan. Sedangkan pada praktiknya,\textit{web scraping} mencakup beragam teknik dan teknologi pemrograman, seperti analisis data dan keamanan informasi.

\textbf{Mengapa Web Scraping?}
\par Ada banyak alasan mengapa web scraping semakin dibutuhkan pada abad ke 21 ini. Dengan semakin berkembangnya data, jumlah data yang tersedia mungkin sudah tidak terhitung lagi. Bayangkan jika kita membutuhkan data-data itu lalu Anda harus mengumpulkan dan menyimpan jutaan data dalam satu file. Teknik web scraping bisa membantu kita untuk mengumpulkan data dengan lebih cepat dan otomatis, semua akan berjalan lancar selama server masih berfungsi.

Efisiennya teknik web scraping ini juga membantu proses pengambilan data demi kebutuhan analisa. Karena web scraping membantu mengumpulkan semua data tanpa terlewat. Dengan begitu, Anda akan mendapat insight yang bernilai dengan lebih cepat. Kita juga bisa memanfaatkan web scraping untuk mengumpulkan data lain yang penting.
