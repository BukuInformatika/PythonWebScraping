%%%%%%%%%%%%%%
%% Run LaTeX on this file several times to get Table of Contents,
%% cross-references, and citations.

%% If you have font problems, you may edit the w-bookps.sty file
%% to customize the font names to match those on your system.

%% w-bksamp.tex. Current Version: Feb 16, 2012
%%%%%%%%%%%%%%%%%%%%%%%%%%%%%%%%%%%%%%%%%%%%%%%%%%%%%%%%%%%%%%%%
%
%  Sample file for
%  Wiley Book Style, Design No.: SD 001B, 7x10
%  Wiley Book Style, Design No.: SD 004B, 6x9
%
%
%  Prepared by Amy Hendrickson, TeXnology Inc.
%  http://www.texnology.com
%%%%%%%%%%%%%%%%%%%%%%%%%%%%%%%%%%%%%%%%%%%%%%%%%%%%%%%%%%%%%%%%

%%%%%%%%%%%%%
% 7x10
%\documentclass{wileySev}

% 6x9
\documentclass{wileySix}

\usepackage{graphicx}
\usepackage{listings}




\usepackage{color}

 
\definecolor{codegreen}{rgb}{0,0.6,0}
\definecolor{codegray}{rgb}{0.5,0.5,0.5}
\definecolor{codepurple}{rgb}{0.58,0,0.82}
\definecolor{backcolour}{rgb}{0.95,0.95,0.92}
 
\lstdefinestyle{mystyle}{
    backgroundcolor=\color{backcolour},   
    commentstyle=\color{codegreen},
    keywordstyle=\color{magenta},
    numberstyle=\tiny\color{codegray},
    stringstyle=\color{codepurple},
    basicstyle=\footnotesize,
    breakatwhitespace=false,         
    breaklines=true,                 
    captionpos=b,                    
    keepspaces=true,                 
    numbers=left,                    
    numbersep=5pt,                  
    showspaces=false,                
    showstringspaces=false,
    showtabs=false,                  
    tabsize=2,
    language=sh
}
 
\lstset{ 
numberstyle=\small, 
numbersep=8pt, 
frame = single, 
language=Python, 
framexleftmargin=10pt}


%%%%%%%
%% for times math: However, this package disables bold math (!)
%% \mathbf{x} will still work, but you will not have bold math
%% in section heads or chapter titles. If you don't use math
%% in those environments, mathptmx might be a good choice.

% \usepackage{mathptmx}

% For PostScript text
\usepackage{w-bookps}

%%%%%%%%%%%%%%%%%%%%%%%%%%%%%%%%%%%%%%%%%%%%%%%%%%%%%%%%%%%%%%%%
%% Other packages you might want to use:

% for chapter bibliography made with BibTeX
% \usepackage{chapterbib}

% for multiple indices
% \usepackage{multind}

% for answers to problems
% \usepackage{answers}

%%%%%%%%%%%%%%%%%%%%%%%%%%%%%%
%% Change options here if you want:
%%
%% How many levels of section head would you like numbered?
%% 0= no section numbers, 1= section, 2= subsection, 3= subsubsection
%%==>>
\setcounter{secnumdepth}{3}

%% How many levels of section head would you like to appear in the
%% Table of Contents?
%% 0= chapter titles, 1= section titles, 2= subsection titles, 
%% 3= subsubsection titles.
%%==>>
\setcounter{tocdepth}{2}

%% Cropmarks? good for final page makeup
%% \docropmarks

%%%%%%%%%%%%%%%%%%%%%%%%%%%%%%
%
% DRAFT
%
% Uncomment to get double spacing between lines, current date and time
% printed at bottom of page.
% \draft
% (If you want to keep tables from becoming double spaced also uncomment
% this):
% \renewcommand{\arraystretch}{0.6}
%%%%%%%%%%%%%%%%%%%%%%%%%%%%%%

%%%%%%% Demo of section head containing sample macro:
%% To get a macro to expand correctly in a section head, with upper and
%% lower case math, put the definition and set the box 
%% before \begin{document}, so that when it appears in the 
%% table of contents it will also work:

\newcommand{\VT}[1]{\ensuremath{{V_{T#1}}}}

%% use a box to expand the macro before we put it into the section head:

\newbox\sectsavebox
\setbox\sectsavebox=\hbox{\boldmath\VT{xyz}}

%%%%%%%%%%%%%%%%% End Demo


\begin{document}


\booktitle{Python Web Scraping
}
%\subtitle{}

\authors{Rolly M. Awangga\\
\affil{Informatics Research Center}
%Floyd J. Fowler, Jr.\\
%\affil{University of New Mexico}
}

\offprintinfo{Cerdas Menguasai Git, First Edition}{Rolly M. Awangga}

%% Can use \\ if title, and edition are too wide, ie,
%% \offprintinfo{Survey Methodology,\\ Second Edition}{Robert M. Groves}

%%%%%%%%%%%%%%%%%%%%%%%%%%%%%%
%% 
\halftitlepage

\titlepage


\begin{copyrightpage}{2019}
%Survey Methodology / Robert M. Groves . . . [et al.].
%\       p. cm.---(Wiley series in survey methodology)
%\    ``Wiley-Interscience."
%\    Includes bibliographical references and index.
%\    ISBN 0-471-48348-6 (pbk.)
%\    1. Surveys---Methodology.  2. Social 
%\  sciences---Research---Statistical methods.  I. Groves, Robert M.  II. %
%Series.\\
%
%HA31.2.S873 2007
%001.4'33---dc22                                             2004044064
\end{copyrightpage}

\dedication{`Jika Kamu tidak dapat menahan lelahnya belajar, 
Maka kamu harus sanggup menahan perihnya Kebodohan.'
~Imam Syafi'i~}

\begin{contributors}
\name{Rolly Maulana Awangga,} Informatics Research Center., Politeknik Pos Indonesia, Bandung,
Indonesia



\end{contributors}

\contentsinbrief
\tableofcontents
\listoffigures
\listoftables
\lstlistoflistings


\begin{foreword}
Sepatah kata dari Kaprodi, Kabag Kemahasiswaan dan Mahasiswa
\end{foreword}

\begin{preface}
Buku ini diciptakan bagi yang awam dengan git sekalipun.

\prefaceauthor{R. M. Awangga}
\where{Bandung, Jawa Barat\\
Februari, 2019}
\end{preface}


\begin{acknowledgments}
Terima kasih atas semua masukan dari para mahasiswa agar bisa membuat buku ini 
lebih baik dan lebih mudah dimengerti.

Terima kasih ini juga ditujukan khusus untuk team IRC yang 
telah fokus untuk belajar dan memahami bagaimana buku ini mendampingi proses 
Intership.
\authorinitials{R. M. A.}
\end{acknowledgments}

\begin{acronyms}
\acro{ACGIH}{American Conference of Governmental Industrial Hygienists}
\acro{AEC}{Atomic Energy Commission}
\acro{OSHA}{Occupational Health and Safety Commission}
\acro{SAMA}{Scientific Apparatus Makers Association}
\end{acronyms}

\begin{glossary}
\term{git}Merupakan manajemen sumber kode yang dibuat oleh linus torvald.

\term{bash}Merupakan bahasa sistem operasi berbasiskan *NIX.

\term{linux}Sistem operasi berbasis sumber kode terbuka yang dibuat oleh Linus Torvald
\end{glossary}

\begin{symbols}
\term{A}Amplitude

\term{\hbox{\&}}Propositional logic symbol 

\term{a}Filter Coefficient

\bigskip

\term{\mathcal{B}}Number of Beats
\end{symbols}

\begin{introduction}

%% optional, but if you want to list author:

\introauthor{Rolly Maulana Awangga, S.T., M.T.}
{Informatics Research Center\\
Bandung, Jawa Barat, Indonesia}

Pada era disruptif  \index{disruptif}\index{disruptif!modern} 
saat ini. git merupakan sebuah kebutuhan dalam sebuah organisasi pengembangan perangkat lunak.
Buku ini diharapkan bisa menjadi penghantar para programmer, analis, IT Operation dan Project Manajer.
Dalam melakukan implementasi git pada diri dan organisasinya.

Rumusnya cuman sebagai contoh aja biar keren\cite{awangga2018sampeu}.

\begin{equation}
ABC {\cal DEF} \alpha\beta\Gamma\Delta\sum^{abc}_{def}
\end{equation}

\end{introduction}

%%%%%%%%%%%%%%%%%%Isi Buku_

\chapter{Permulaan}
Pada bab ini, kita akan menjalankan program Python yang pertama yaitu, hello\_world.py. Tahap pertama, kita akan memastikan apakah Python sudah terinstall dengan benar. Kita juga akan menginstall teks editor untuk menulis program Python.
\section{Menyiapkan \textit{Programming Environment}}
Di sini, kita akan melihat dua versi utama Python yang saat ini digunakan dan menguraikan langkah-langkah untuk menyiapkan Python pada sistem.
\subsection{Python 2 dan Python 3}

Saat ini, telah tersedia dua versi Python: Python 2 dan Python yang lebih baru, yaitu Python 3. Mengapa? Karena setiap bahasa pemrograman berevolusi ketika ide dan teknologi baru muncul atau berkembang, dan tentu saja para pengembang bahasa Python terus membuat Python agar lebih fleksibel dan kuat. Sebagian besar perubahan yang dilakukan, berkembang sedikit demi sedikit secara teratur dan hampir tidak terlihat, tetapi dalam beberapa kasus kode yang ditulis untuk Python 2 mungkin tidak berjalan dengan baik pada sistem yang menggunakan Python 3.
\subsection{Menjalankan Kode}

Python dilengkapi dengan \textit{interpreter} yang berjalan di jendela terminal, yang memungkinkan kita untuk mencoba kode Python tanpa harus menyimpan dan menjalankan seluruh program. Contohnya adalah seperti pada
\begin{lstlisting}[language=Python, label={lst:hello}, caption=Kode Pada Jendela Terminal]
>>> print("Hello World!")
Hello World!
\end{lstlisting}
Pada baris pertama, adalah baris yang kita tuliskan sendiri perintahnya, lalu bisa dieksekusi dengan cara menekan tombol enter. Sebagian besar contoh yang akan ditampilkan pada buku ini akan dijalankan melalui teks editor, karena kita akan menulis kode pada teks editor tersebut. Setiap kali Anda melihat tiga buah tanda tanda panah sebelah kiri seperti pada listing \ref{lst:hello} itu artinya kita sedang mengunakan jendela terminal. Listing \ref{lst:hello} menunjukan program sederhana yang umum dilakukan \textit{programmer} pada awal pembelajaran. Jika kode berjalan dengan sesuai, maka apapun program yang telah dibuat menggunakan Python bisa berjalan dengan sempurna.
\subsection{Python pada Berbagai Sistem Operasi}

Python adalah bahasa pemrograman lintas platform, yang artinya berjalan pada semua sistem operasi, seperti Windows, Linux, dan OS X. Program Python apa pun bisa dijalankan pada seluruh sitem yang telah terpasang Python sebelumnya. Namun, cara untuk mengatur Python pada setiap sistem operasi akan berbeda. Di bagian ini kita akan belajar cara men-\textit{set up} Python dan menjalankan program Hello World pada setiap sistem operasi.Yang pertama adalah, kita akan memeriksa apakah Python telah ter-\textit{install} dengan benar atau belum. Kemudian, menginstal teks editor sederhana dan menyimpan file Python kosong bernama hello\_world.py. Tahap terakhir, yaitu menjalankan program Hello World.
\subsubsection{Python pada Windows}

Pertama, periksa apakah Python telah terpasang pada sistem atau belum, dengan cara buka \textit{Command Prompt} dengan memasukkan atau mengetikan kata "cmd" pada menu Start. Pada jendela terminal, ketikan python dalam huruf kecil. Jika mendapatkan prompt Python \verb|(>>>)|, Python telah terpasang pada sistem. Namun, jika melihat pesan kesalahan yang mengatakan bahwa python adalah  bukan perintah yang dikenali.
Jika demikian, unduh Python \textit{Installer} untuk Windows. Buka \verb|http://python.org/downloads/|. Akan tersedia dua pilihanl, satu untuk mengunduh Python 3 dan satu lagi untuk mengunduh Python 2. Klik tombol Python 3, lalu secara otomatis mulai mengunduh \textit{Installer}. Setelah \textit{Installer} berhasil terunduh, jalankan \textit{Installer}. Pastikan mencentang opsi \textit{Add Python to PATH}, agar lebih mudah untuk mengkonfigurasi sistem dengan benar.

Sekarang kita akan memulai untuk mencoba apakah Python sudah terpasang dengan benar atau belum, dengan cara membuka \textit{Command Prompt} lalu mengetik \textbf{python}. Jika mendapatkan prompt Python \verb|(>>>)|, Python telah terpasang pada sistem dengan berhasil. 
\begin{lstlisting}[language=Python, label={lst:success}, caption=Python Yang Telah Terpasang]
C:\> python
Python 3.5.0 (v3.5.0:374f501f4567,
Mar 19 2019, 22:15:05)
[MSC v.1900 32 bit (Intel)] on win32
Type "help", "copyright", "credits" or "license"
for more information.
>>>
\end{lstlisting}

\subsection{Mengatasi Masalah Umum Pada Python}
Jika tidak dapat menjalankan hello\_world.py, berikut adalah beberapa solusi yang dapat gunakan atau dapat dicoba:
\begin{itemize}
\item Ketika sebuah program mengandung kesalahan yang cukup besar, Python akan selalu menampilkan \textit{traceback}. Python melihat melalui file dan mencoba melaporkan atau memberitahu masalahnya. \textit{Traceback} bisa memberi petunjuk tentang masalah apa yang menghambat program berjalan.
\item  Ingat bahwa sintaks sangat penting dalam pemrograman, tanda kutip yang tidak cocok, atau tanda kurung yang tidak cocok dapat menghambat program berjalan.
\end{itemize}


\chapter{Mengenal Web Scraping}
\section{Web Scraping}
\subsection{Apa Itu Web Scraping?}

Secara teori, \textit{web scraping} adalah praktik mengumpulkan data melalui cara apa pun selain program yang berinteraksi dengan API (atau melalui manusia menggunakan browser). Cara ini paling umum dilakukan dengan menulis program otomatis yang membutuhkan server web, meminta data (biasanya dalam bentuk HTML, Json, dan file lain yang terdiri dari halaman web), dan kemudian mem-\textit{parsing} data tersebut untuk mengekstrak informasi yang diperlukan. Sedangkan pada praktiknya,\textit{web scraping} mencakup beragam teknik dan teknologi pemrograman, seperti analisis data dan keamanan informasi.

\subsection{Mengapa Web Scraping?}

Ada banyak alasan mengapa web scraping semakin dibutuhkan pada abad ke 21 ini. Dengan semakin berkembangnya data, jumlah data yang tersedia mungkin sudah tidak terhitung lagi. Bayangkan jika kita membutuhkan data-data itu lalu Anda harus mengumpulkan dan menyimpan jutaan data dalam satu file. Teknik web scraping bisa membantu kita untuk mengumpulkan data dengan lebih cepat dan otomatis, semua akan berjalan lancar selama server masih berfungsi.

Efisiennya teknik web scraping ini juga membantu proses pengambilan data demi kebutuhan analisa. Karena web scraping membantu mengumpulkan semua data tanpa terlewat. Dengan begitu, Anda akan mendapat insight yang bernilai dengan lebih cepat. Kita juga bisa memanfaatkan web scraping untuk mengumpulkan data lain yang penting.

Selain di dunia bisnis, di dunia seni pun, web scraping telah diterapkan untuk proyek 2006 “We Feel Fine” oleh Jonathan Harris dan Sep Kamvar, men-\textit{scrap} berbagai situs blog berbahasa Inggris untuk frasa yang dimulai dengan “I feel” atau “I am feeling” dan dengan data tersebut, bisa diolah menjadi visualisasi data bagaimana perasaan orang-orang di dunia setiap harinya. Terlepas dari bidang apapun, hampir selalu ada cara\textit{web scraping} dapat memandu bisnis lebih efektif dan meningkatkan produktivitas.

\subsection{Teknik web scraping}
Karena web scraping sudah mulai familiar, maka banyak orang yang melakukannya karena beberapa kemudahan yang telah dijabarkan diatas, ada beberapa teknik automasi yang bisa kita lakukan untuk melakukan web scraping.
\begin{enumerate}

\item Parsing HTML

Parsing HTML adalah salah satu teknik yang paling banyak digunakan dalam web parsing atau web scraping. Biasanya parsing HTML dilakukan menggunakan JavaScript dan menarget halaman HTML linear dan nested. Metode ini dapat mengidentifikasi script HTML dari websia. Script ini juga kemudian digunakan untuk mengekstraksi text, links, dan data.

\item Parsing DOM

\textit{Content, style, dan XML structure} didefinisikan dalam Document Object Model atau yang biasa disebut DOM. Beberapa programmer yang ingin mengetahui cara kerja sebuah internal web dan ingin mengekstrak \textit{script} yang berjalan di dalamnya akan lebih memilih untuk melakukan \textit{web scraping} melalui parsing DOM. Node dikumpulkan terlebih dahulu menggunakan DOM yang terlah diparsing dan XPath membantu mempermudah proses scraping.

\item XPath

\textit{XML Path Language} atau lebih familiar dengan istilah XPath adalah bahasa \textit{query} yang bekerja pada dokumen Extensible Markup Language atau biasa disebut XML. Dokumen XML biasa disusun dengan \textit{tree structure}, XPath bisa digunakan untuk menganalisa struktur dokumen dengan memilih nodes berdasarkan parameter yang telah tersedia. XPath juga lazim digunakan bersamaan dengan DOM parsing dalam mengesktrasi seluruh halaman website.

\item Google Docs

Salah satu produk Google yaitu Google Sheets juga ternyata bisa dipakai sebagai salah satu alat scraping,, dan ini adalah salah satu alat scraping yang cukup popoler karena cara penggunaannya yang tidak rumit dan tidak membutuhkan keahlian khusus. Pada Google Sheets sendiri, kita bisa memanfaatkan fungsi IMPORTXML untuk melakukan scraping data dari website. Selain itu, juga bisa menggunakan command ini untuk melihat apakah sebuah website aman dari tindakan scraping atau tidak.
\end{enumerate}

\subsection{Resiko dan Ancaman}

Pada perspektif Bisnis, otomatisasi web seperti web scraping, juga memiliki dampak negatif. Yang berpengaruh disini adalah reputasi perusahaan, SOP, dan proses bisnis internal. Beberapa resiko yang mungkin akan timbul adalah:
\begin{enumerate}

\item Data statistik: Setiap \textit{request} yang dilakukan oleh robot sangat kecil kemungkinannya akan tercatat pada laporan statistik, sehingga akan menyebabkan data analisis tersebut akan menjadi bias atau tidak akurat. Dan dengan ketidak akuratannya data statistik tersebut, tim bisnis marketing akan beresiko salah dalam contohnya menganalisa pasar, dan mengambil keputusan bisnis.

\item Bulk Order: Dengan otomatis, web robot memungkinkan untuk membuat random dan distributed order fiktif. Tujuannya bisa beragam, mungkin ingin merusak bisnis kompetitor dengan menghabiskan stock barang yang dimilikinya, atau hanya sekedar seseorang yang mengambil barang promosi dengan jumlah sangat banyak, dan menjuanya kembali dengan harga normal.
\end{enumerate}

\subsection{Pertahanan}
Pertahanan yang paling efektif dalam menaggulangi bot adalah memblokir IP agar tidak dapat mengakses website lagi, atau bisa dengan cara me-\textit{redirect}-nya ke halaman captcha, seperti yang dilakukan google.com apabila mereka mencurigai suatu IP Penggunaan Captcha sangat efektif dalam mendeteksi apakah yang mengakses website adalah user sesungguhnya atau hanya robot.

\chapter{Membangun Web Scraper}
\section{Membangun Web Scraper}
\subsection{BeautifulSoup}
\par Beautiful Soup adalah \textit{library} Python untuk menarik atau mengambil data dari file HTML dan XML. Hal ini bisa bekerja dengan parser apapun untuk menavigasi, mencari, dan memodifikasi pohon parse. Ini biasanya menghemat waktu menulis kode dan hari kerja. BeautifulSoup juga mencoba memahami yang tidak masuk akal; sedikit membantu memformat ulang dan mengatur web yang berantakan dengan memperbaiki HTML yang kacau dan menyajikan kepada kita struktur dalam Python yang mudah dibaca dan mewakili struktur XML.
\textbf{Menginstal BeautifulSoup}
Karena \textit{library} BeautifulSoup bukan Python \textit{library} default, maka harus terlebih dahulu diinstal. Kita akan menggunakan \textit{library} BeautifulSoup versi 4 (BS4).

untuk Linux:

\$ sudo apt-get install python-bs4

dan untuk Mac:

\$ sudo easy\_install pip

Menginstal \textit{package} di Windows hampir sama dengan proses untuk Mac dan Linux. Unduh rilisan terakhir BeautifulSoup 4, navigasikan ke direktori tempat megekstrak lalu jalankan:

instal setup.py python

BeautifulSoup sekarang akan dikenali sebagai \textit{library} Python di sistem. Kita masih bisa menguji ini dengan membuka terminal Python dan mengimportnya:

\begin{algorithm}

\$ python
bs4 import from BeautifulSoup
\end{algorithm}


\textbf{Menjalankan BeautifulSoup}

Sebelum kita melompat pada kode, mari kita bahas terlebih dahulu hal-hal mendasar dari HTML, dan beberapa aturan dalam \textit{web scraping}
\begin{enumerate}
\item HTML TAG
\begin{algorithm}
<!DOCTYPE html>  
<html>  
    <head>
    </head>
    <body>
        <h1> First Scraping </h1>
        <p> Hello World </p>
    <body>
</html>
\end{algorithm}
\begin{enumerate}

\item \verb|<! DOCTYPE html>:| Dokumen HTML harus dimulai dengan deklarasi tipe.
\item Dokumen HTML terdapat di antara \verb|<html>| dan \verb|</html>.|
\item Deklarasi meta dan skrip dari dokumen HTML berada di antara \verb|<head>| dan \verb|</head>.|
\item Bagian yang terlihat dari dokumen HTML adalah antara tag \verb|<body>| dan \verb|</body>.|
\item Judul judul didefinisikan dengan tag \verb|<h1>| hingga \verb|<h6>.|
\item Paragraf didefinisikan dengan tag \verb|<p>.|
\end{enumerate}

Tag berguna lainnya termasuk \verb|<a>| untuk hyperlink, \verb|<table>| untuk tabel, \verb|<tr>| untuk baris tabel, dan \verb|<td>| untuk kolom tabel. Juga, tag HTML kadang-kadang datang dengan atribut id atau kelas. Atribut id menentukan id unik untuk tag HTML dan nilainya harus unik dalam dokumen HTML. Atribut kelas digunakan untuk menentukan gaya yang sama untuk tag HTML dengan kelas yang sama. Kita dapat menggunakan id dan kelas ini untuk membantu kita menemukan data yang kita inginkan.

\item{Aturan Scraping}
\begin{enumerate}
\item Periksa terllebih dahulu Syarat dan Ketentuan situs web tujuan sebelum melakukan \textit{scraping}. Berhati-hatilah untuk membaca pernyataan tentang penggunaan data secara legal. 
\item Jangan meminta data dari situs web terlalu agresif atau terlalu sering dengan program yang telah dibuat (juga dikenal sebagai spam), karena ini dapat merusak situs web. Pastikan program berperilaku dengan cara yang masuk akal (mis. Bertindak seperti manusia). Satu request untuk satu halaman web per detik adalah yang paling disarankan.
\item \textit{layout} situs web dapat berubah dari waktu ke waktu, jadi pastikan untuk mengunjungi kembali situs dan menulis ulang kode sesuai kebutuhan.
\end{enumerate}
\end{enumerate}



\chapter{BeautifulSoup}
Find () dan findAll () adalah dua fungsi BeautifulSoup yang akan sering digunakan. Dengan kedua fungsi ini, kita dapat dengan mudah mem-\textit{filter} halaman HTML untuk menemukan daftar tag yang diinginkan, atau satu tag, berdasarkan berbagai atribut mereka. Kedua fungsi ini sangat mirip:

\verb|findAll (tag, atribut, rekursif, teks, batas, kata kunci)|
\verb|find (tag, atribut, rekursif, teks, kata kunci)|

Dalam banyak kesempatan, 95\% dari waktu kita saat menulis kode, kita akan menggunakan dua argumen pertama, yaitu tag dan attribut. Tag digunakan untuk mendeklarasikan sebuah objek.  Misalnya, yang berikut ini akan mengembalikan daftar semua argumen tag dalam dokumen: 1

\verb|.findAll ({"h1", "h2", "h3", "h4", "h5", "h6"})|

Argumen atribut mengambil library Python dari atribut dan mencocokkan tag yang ada padai salah satu dari atribut tersebut. Misalnya, fungsi berikut akan me-\textit{return} tag hijau dan merah di dokumen HTML:

\verb|.findAll ("span", {"class": "green", "class": "red"})|

Argumen rekursif adalah boolean. Seberapa dalam kita ingin masuk ke dalam sebuah struktur HTML? Jika rekursif diset ke True, fungsi findAll melihat ke \textit{children}, dan \textit{children’s children}, untuk tag yang cocok dengan parameter. Jika itu salah, ia hanya akan melihat tag paling atas dalam dokumen. Secara default, findAll bekerja secara rekursif (rekursif diset ke True);


\bibliographystyle{IEEEtran} 
%\def\bibfont{\normalsize}
\bibliography{references}


%%%%%%%%%%%%%%%
%%  The default LaTeX Index
%%  Don't need to add any commands before \begin{document}
\printindex

%%%% Making an index
%% 
%% 1. Make index entries, don't leave any spaces so that they
%% will be sorted correctly.
%% 
%% \index{term}
%% \index{term!subterm}
%% \index{term!subterm!subsubterm}
%% 
%% 2. Run LaTeX several times to produce <filename>.idx
%% 
%% 3. On command line, type  makeindx <filename> which
%% will produce <filename>.ind 
%% 
%% 4. Type \printindex to make the index appear in your book.
%% 
%% 5. If you would like to edit <filename>.ind 
%% you may do so. See docs.pdf for more information.
%% 
%%%%%%%%%%%%%%%%%%%%%%%%%%%%%%

%%%%%%%%%%%%%% Making Multiple Indices %%%%%%%%%%%%%%%%
%% 1. 
%% \usepackage{multind}
%% \makeindex{book}
%% \makeindex{authors}
%% \begin{document}
%% 
%% 2.
%% % add index terms to your book, ie,
%% \index{book}{A term to go to the topic index}
%% \index{authors}{Put this author in the author index}
%% 
%% \index{book}{Cows}
%% \index{book}{Cows!Jersey}
%% \index{book}{Cows!Jersey!Brown}
%% 
%% \index{author}{Douglas Adams}
%% \index{author}{Boethius}
%% \index{author}{Mark Twain}
%% 
%% 3. On command line type 
%% makeindex topic 
%% makeindex authors
%% 
%% 4.
%% this is a Wiley command to make the indices print:
%% \multiprintindex{book}{Topic index}
%% \multiprintindex{authors}{Author index}

\end{document}

